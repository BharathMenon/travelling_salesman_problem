\documentclass{article}
\usepackage{graphicx} % Required for inserting images
\usepackage{xcolor}

\title{Travelling Salesman Problem}
\author{Tanmay Deshpande and Bharat Menon}
\date{February 2023}
\definecolor{base}{HTML}{1e1e2e }
\definecolor{text}{HTML}{cdd6f4}
\begin{document}
    \pagecolor{base}
    \color{text}
\maketitle

\section{Introduction}
The travelling salesman problem asks the following question:
"Given a list of cities and the distances between each pair of cities,
what is the shortest possible route that visits each city exactly once
and returns to the origin city?"
The objective of this project is to find the shortest possible path
among a set of given points (along with their coordinates) that
traverses every single point exactly once, and then returns back to the
starting point, thus attempting to solve the travelling salesman
problem.
This is done with the help of an algorithm, used to find the optimal
path. The algorithm is explained hand-in-hand with a live example as
that will be much easier to understand.

\section{Working of the Algorithm}
\begin{enumerate}

    \item We assume that n distinct points are fed by the user to the
algorithm, along with their coordinates.

    \item Then, we create an nXn matrix which contains the
distances from each point to every other point as the
elements. (As in, the element M [a, b], where M is the
matrix, contains the distance of a from b.) Note that
distance from one point to itself is considered as infinity,
hence an entry of M [i, i] is considered infinity for all i in
n. A set of values in the example case has been taken and
formed into a matrix as shown here

$$
\left[ {\begin{array}{ccccc}
     \infty & 20 & 30 & 10 & 11\\
     15 & \infty & 16 & 4 & 2\\
     3 & 5 & \infty & 2 & 4\\
     19 & 6 & 18 & \infty & 3\\
     16 & 4 & 7 & 16 & \infty\\
\end{array} }
\right]
$$

    \item Now, we “reduce” this matrix. This is done by finding the
minimum element from across each row and subtracting
them from each of the elements in the corresponding row.
Then we repeat this for every column. The sum of all the
removed elements is called the reduced cost. The
resultant matrix achieved is the first reduced matrix. It
shall be referred to as C. The reduced cost as calculated is 25. The reduced matrix has been
shown here

$$
\left[ {\begin{array}{ccccc}
     \infty & 10 & 17 & 0 & 1\\
     12 & \infty & 11 & 2 & 0\\
     0 & 3 & \infty & 0 & 2\\
     15 & 3 & 12 & \infty & 0\\
     11 & 0 & 0 & 12 & \infty\\
\end{array} }
\right]
$$

    \item Now, it is assumed that the path starts from one point. In
this case, we consider point 1 to be the starting point
(With regards to length of path, starting point won’t
matter).

    \item First, consider the path $1 \rightarrow 2$. Now, to find the cost of
node (or point) 2, we need to reduce the first reduced
matrix in a particular way:

    \begin{enumerate}

        \item First, make all the elements in the row of the
starting point and the column of the ending
point as infinity. So, all the elements in the row
1 and column 2 are made infinity.

        \item Then, as once we come from a point to another,
it can’t go back, the element of the order
(from\_point, to\_point) must also be set to
infinity. In this case, C (1,2) is set to infinity.

        \item The reduced matrix for path $1 \rightarrow 2$ is given here

        $$
        \left[ {\begin{array}{ccccc}
             \infty & \infty & \infty & \infty & \infty\\
             \infty & \infty & 11 & 2 & 0\\
             0 & \infty & \infty & 0 & 2\\
             15 & \infty & 12 & \infty & 0\\
             11 & \infty & 0 & 12 & \infty\\
        \end{array} }
        \right]
        $$


        \item Finally, the reduced cost for this node shall be
calculated according to the formula:
        $$M[ from\_node, to\_node] + r$$
where r is the
reduced cost of the current matrix.
So, reduced cost for node 2 in this example is:
10 + 0 = 10.

    \end{enumerate}

    \item Similarly, the reduced matrix and the corresponding
cost of reduction is found for all the points from the
first node. In our example, the reduced costs for node
3,4 and 5 are 27, 0 and 6 respectively.

    \item Then, the node with least cost of reduction is found,
and that is assumed to be the next node in the optimal
path. Hence, in our example, the point 4 will be next
node.

    \item Now, from the next node, the possible paths are further
extended to possible points. Then, just like in the
previous case, reduced matrixes are taken cost of
reductions are found.
    \begin{enumerate}
        \item From node 4, the path can go to nodes 2 or 3.
This matrix is then reduced, in similar manner
as the previous case. The cost of reduction here
is C (from\_node, to\_node) + r, where r is the
cost of reduction of to\_node.
        \item For the path $4 \rightarrow 2$, The elements of the 4th
row and the 2nd column are turned to infinity.
The elements C(2,1) and C(2,4) are also turned
to infinity. Then, the cost of reduction is found.
The cost of reduction of node 4 to node 2,3 and
5 happens to be 3, 25 and 11 respectively.
    \end{enumerate}

    \item Again, the node with the least cost of reduction is
chosen as the next point in the path. Hence, the path
now becomes: $1 \rightarrow 4 \rightarrow 2$.
    \item Next, the similar method is followed to find the next
node (from 3 and 5), by reducing the matrix and hence
finding the cost of reduction for the remaining points.
    \item So, the cost of reduction for node 3 and 5 respectively.
    \item Hence, the next node would then be 5 and the node
after that automatically becomes 3.
    \item The path eventually returns to where it started, so it
returns to 1.

\end{enumerate}

\section{Conclusion}
Hence, the most optimal path according to our
algorithm is:
$1 \rightarrow 4 \rightarrow 2 \rightarrow 5 \rightarrow 3 \rightarrow 1$.
In this way, the algorithm is able to provide the
optimal path for n number of inputted points

\end{document}
